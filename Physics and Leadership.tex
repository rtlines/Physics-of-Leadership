
\documentclass{article}
%%%%%%%%%%%%%%%%%%%%%%%%%%%%%%%%%%%%%%%%%%%%%%%%%%%%%%%%%%%%%%%%%%%%%%%%%%%%%%%%%%%%%%%%%%%%%%%%%%%%%%%%%%%%%%%%%%%%%%%%%%%%%%%%%%%%%%%%%%%%%%%%%%%%%%%%%%%%%%%%%%%%%%%%%%%%%%%%%%%%%%%%%%%%%%%%%%%%%%%%%%%%%%%%%%%%%%%%%%%%%%%%%%%%%%%%%%%%%%%%%%%%%%%%%%%%
\usepackage{amsfonts}
\usepackage{amsmath}
\usepackage{url}

\setcounter{MaxMatrixCols}{10}
%TCIDATA{OutputFilter=LATEX.DLL}
%TCIDATA{Version=5.50.0.2953}
%TCIDATA{<META NAME="SaveForMode" CONTENT="1">}
%TCIDATA{BibliographyScheme=BibTeX}
%TCIDATA{Created=Sunday, January 01, 2017 17:13:56}
%TCIDATA{LastRevised=Thursday, June 14, 2018 14:50:32}
%TCIDATA{<META NAME="GraphicsSave" CONTENT="32">}
%TCIDATA{<META NAME="DocumentShell" CONTENT="Standard LaTeX\Standard LaTeX Article">}
%TCIDATA{Language=American English}
%TCIDATA{CSTFile=40 LaTeX article.cst}

\newtheorem{theorem}{Theorem}
\newtheorem{acknowledgement}[theorem]{Acknowledgement}
\newtheorem{algorithm}[theorem]{Algorithm}
\newtheorem{axiom}[theorem]{Axiom}
\newtheorem{case}[theorem]{Case}
\newtheorem{claim}[theorem]{Claim}
\newtheorem{conclusion}[theorem]{Conclusion}
\newtheorem{condition}[theorem]{Condition}
\newtheorem{conjecture}[theorem]{Conjecture}
\newtheorem{corollary}[theorem]{Corollary}
\newtheorem{criterion}[theorem]{Criterion}
\newtheorem{definition}[theorem]{Definition}
\newtheorem{example}[theorem]{Example}
\newtheorem{exercise}[theorem]{Exercise}
\newtheorem{lemma}[theorem]{Lemma}
\newtheorem{notation}[theorem]{Notation}
\newtheorem{problem}[theorem]{Problem}
\newtheorem{proposition}[theorem]{Proposition}
\newtheorem{remark}[theorem]{Remark}
\newtheorem{solution}[theorem]{Solution}
\newtheorem{summary}[theorem]{Summary}
\newenvironment{proof}[1][Proof]{\noindent\textbf{#1.} }{\ \rule{0.5em}{0.5em}}
\input{tcilatex}
\begin{document}

\title{Physics and Leadership: Managing Chaos}
\author{Todd Lines, Steven Turcotte, Brian Pyper, Brian Tonks \\
%EndAName
Brigham Young University}
\date{February 22, 2015}
\maketitle

\begin{abstract}
The principles of physics have been applied to social settings for
generations. Many organization's structure and behavior are patterned after
the deterministic models based on Newtonian physics. These mechanistic
models work well when leading teams that build mechanistic products. But in
physics, the deterministic models of Newton have given way to the less
deterministic disciplines of Chaos theory and Quantum Mechanics. This paper
addresses the application of non-deterministic physics ideas in the area of
leadership. An increased emphasis on relationships as opposed to rigid job
functions and a need for constant feedback in the organization are found to
be the results of applying current physical understanding to leadership. A
model of leading with constant feedback, much like a forecast model in
weather prediction, is suggested.
\end{abstract}

\section{Physics}

In the country side of Europe, straddling the Franco-Swiss boarder near Lake
Geneva lies an engineering and scientific marvel that rivals the Great
Pyramids.\FRAME{dtbpFU}{2.755in}{2.0702in}{0pt}{\Qcb{CERN Large Hadron
Collider location (https://cds.cern.ch/record/1295244)}}{}{Figure}{\special%
{language "Scientific Word";type "GRAPHIC";maintain-aspect-ratio
TRUE;display "USEDEF";valid_file "T";width 2.755in;height 2.0702in;depth
0pt;original-width 9.9912in;original-height 7.491in;cropleft "0";croptop
"1";cropright "1";cropbottom "0";tempfilename
'OJ4Q810Y.wmf';tempfile-properties "XPR";}}

It consists of a $27\unit{km}$ ($16.8\unit{mi}$) circular ring dug $100\unit{%
m}$ below the surface along the underlying bedrock (about the length of a
football field down into the ground). This is the Large Hadron Collider(LHC)
at CERN.\FRAME{dtbpF}{2.9182in}{2.1563in}{0pt}{}{}{Figure}{\special{language
"Scientific Word";type "GRAPHIC";maintain-aspect-ratio TRUE;display
"USEDEF";valid_file "T";width 2.9182in;height 2.1563in;depth
0pt;original-width 10.1503in;original-height 7.4832in;cropleft "0";croptop
"1";cropright "1";cropbottom "0";tempfilename
'OJ4MKX01.wmf';tempfile-properties "XPR";}}

Inside this tunnel, there are $1232$ superconducting electromagnets, each
about $15$ yards long. These magnets are kept at $-271.25\unit{%
%TCIMACRO{\U{2103}}%
%BeginExpansion
{}^{\circ}{\rm C}%
%EndExpansion
}$ or just $1.9$ degrees above absolute zero. The magnets line the circular
tunnel, and through the center of this giant magnetic ring, protons are
accelerated to very near the speed of light. The protons meet with a similar
group of protons that were accelerated the opposite direction. The resulting
collisions form small regions of space with very high energy densities.

In this dense region of energy, new particles are formed. The designated
meeting place for the two sets of speeding proton is a two story tall
detector. The detectors also contain superconducting magnets kept at $-270%
\unit{%
%TCIMACRO{\U{2103}}%
%BeginExpansion
{}^{\circ}{\rm C}%
%EndExpansion
}$. \FRAME{dtbpF}{3.4175in}{2.3913in}{0pt}{}{}{Figure}{\special{language
"Scientific Word";type "GRAPHIC";maintain-aspect-ratio TRUE;display
"USEDEF";valid_file "T";width 3.4175in;height 2.3913in;depth
0pt;original-width 5.8254in;original-height 4.0672in;cropleft "0";croptop
"1";cropright "1";cropbottom "0";tempfilename
'OJ4MNV02.wmf';tempfile-properties "XPR";}}

The detectors can follow the tiny particles that result from the collisions.
All of this, thought, equipment, and effort is designed to answer a simple
question, \textquotedblleft What is mass?\textquotedblright

You are probably familiar with the idea of mass. It is how much material
there is in an object. If we give you a $10lb$ bag of sugar and a $50lb$ bag
of sugar, you would notice that there is more sugar in the $50lb$ bag. We
would say that the $50lb$ bag has more mass. So mass does not seem too
mysterious.

But to physicists it is very mysterious. Mass has the property of resisting
motion. If you try to throw the $50lb$ bag of sugar, it is harder to move
than the $10lb$ bag; but why? What makes things with mass hard to move? And
if we understood this, could we more easily move objects? This is the sort
of question that physics tackles. The current idea is that there is
something in space, itself, that interacts with objects that have mass. That
something makes it hard for massive objects to move through space. That
something is named the Higgs field. The great Brian Greene says it this way,

\begin{quotation}
Mass is the resistance an object offers to having its speed changed.

You take a baseball. When you throw it, your arm feels resistance. A
shotput, you feel that resistance. The same way for particles. Where does
the resistance come from? And the theory was put forward that perhaps space
was filled with an invisible \textquotedblleft stuff,\textquotedblright\ an
invisible molasses-like \textquotedblleft stuff,\textquotedblright\ and when
the particles try to move through the molasses, they feel a resistance, a
stickiness. It's that stickiness which is where their mass comes from....
That creates the mass\cite{Green}
\end{quotation}

But is this Higgs field really there? And does it really make mass hard to
move? To invent potential new devices based on the ideas of the Higgs field,
we first have to show that such a field exists. But how can you know if a
Higgs field is there?

You must find a testable action that directly relates to the existence of
the Higgs field. In physics this means forming an equation that predicts an
outcome, then testing to see if that outcome happens. \FRAME{dtbpF}{2.1767in%
}{2.0667in}{0pt}{}{}{Figure}{\special{language "Scientific Word";type
"GRAPHIC";maintain-aspect-ratio TRUE;display "USEDEF";valid_file "T";width
2.1767in;height 2.0667in;depth 0pt;original-width 5.3423in;original-height
5.0682in;cropleft "0";croptop "1";cropright "1";cropbottom "0";tempfilename
'PA8EXY06.wmf';tempfile-properties "XPR";}}The equation for the Higgs field
is complicated and understanding the details is not important for our
consideration of physics and leadership. What we should understand is that
this equation, like any other equation in physics, predicts an outcome that
can be tested. In the case of the Higgs field, the equation predicts that if
the energy density in a collision is high enough, occasionally a small
particle will form. This particle is called the \textquotedblleft Higgs
boson.\textquotedblright\ If this particle exists, then there is very likely
a Higgs field, and we can verify our ideal of what mass is. Of course, the
experiment to see if the Higgs boson exists is the experiment for which the
LHC was built. It is an extremely difficult experiment to perform. The
collisions happen so quickly that it takes rooms of computers to monitor and
record what happens.

\FRAME{dtbpFU}{3.2448in}{2.3808in}{0pt}{\Qcb{CERN CMS Detector Computer
Analysis System.}}{}{Figure}{\special{language "Scientific Word";type
"GRAPHIC";maintain-aspect-ratio TRUE;display "USEDEF";valid_file "T";width
3.2448in;height 2.3808in;depth 0pt;original-width 10.4582in;original-height
7.6579in;cropleft "0";croptop "1";cropright "1";cropbottom "0";tempfilename
'OJ4NEG03.wmf';tempfile-properties "XPR";}}

After all the effort, on July 4 of 2012, the Higgs boson was discovered. And
this means that our idea of how mass works may have merit!

What will we build with this new discovery? It is hard to predict. But in
the past, engineers using the ideas tested by physics have built marvelous
things. Even now we have the people at CERN to thank for the World Wide
Web---an invention created to help the thousands of scientists and engineers
communicate and share the experimental data from all over the world to build
and operate the LHC at CERN.\FRAME{dtbpFU}{3.6521in}{2.3791in}{0pt}{\Qcb{%
CERN Processing Facality, the birth place of the World Wide Web}}{}{Figure}{%
\special{language "Scientific Word";type "GRAPHIC";maintain-aspect-ratio
TRUE;display "USEDEF";valid_file "T";width 3.6521in;height 2.3791in;depth
0pt;original-width 9.9912in;original-height 6.4913in;cropleft "0";croptop
"1";cropright "1";cropbottom "0";tempfilename
'OJ4NI304.wmf';tempfile-properties "XPR";}}

But over the past few hundred years the ideas of physics have been building
ever more clever devices that make our life more interesting, from microwave
ovens, to cell phones, to computer systems, to Teflon coated pans.

\FRAME{dtbpFU}{3.1692in}{2.3895in}{0pt}{\Qcb{Physics Spin-offs}}{}{Figure}{%
\special{language "Scientific Word";type "GRAPHIC";maintain-aspect-ratio
TRUE;display "USEDEF";valid_file "T";width 3.1692in;height 2.3895in;depth
0pt;original-width 6.3667in;original-height 4.7919in;cropleft "0";croptop
"1";cropright "1";cropbottom "0";tempfilename
'OJ4NL605.wmf';tempfile-properties "XPR";}}

The idea that became physics has done marvelous things. But how did this

all get started?

\section{The Big Idea: Find Truth by Testing Actions Distilled from Theory}

Let's go back to a time before physics, to the time of the ancient greek
philosophers. Plato and Aristotle.

The study of how things move and interact was called \textquotedblleft
Natural Philosophy\textquotedblright\ and was practiced by thinking about
how things ought to work. For example, if two objects fell, the one with the
most mass ought to fall faster. That was the opinion of Aristotle. And that
opinion held for hundreds of years. If you wanted to know how the universe
worked, you picked up a copy of Aristotle, and read his opinions.

\FRAME{fphFU}{1.2098in}{1.5469in}{0pt}{\Qcb{Plato and Aristole}}{}{Figure}{%
\special{language "Scientific Word";type "GRAPHIC";maintain-aspect-ratio
TRUE;display "USEDEF";valid_file "T";width 1.2098in;height 1.5469in;depth
0pt;original-width 4.12in;original-height 5.2731in;cropleft "0";croptop
"1";cropright "1";cropbottom "0";tempfilename
'PA8F8207.wmf';tempfile-properties "XPR";}}

The ancient philosophers did advance human knowledge. But their ideas
remained untested. The ideas were accepted at truth, but without any
verification.

\subsection{Galileo}

\FRAME{dtbpFU}{1.1903in}{1.8343in}{0pt}{\Qcb{Bust of Galileo from the
Galileo Museum, Florence Italy}}{}{Figure}{\special{language "Scientific
Word";type "GRAPHIC";maintain-aspect-ratio TRUE;display "USEDEF";valid_file
"T";width 1.1903in;height 1.8343in;depth 0pt;original-width
4.875in;original-height 7.542in;cropleft "0";croptop "1";cropright
"1";cropbottom "0";tempfilename 'OJ4NOU07.wmf';tempfile-properties "XPR";}}

Galileo Galilei (1564-1642) is credited with first popularizing the idea of
testing the ideas about how the universe works. To see how this works,
suppose we have two objects, one twice as massive as the other. And further
suppose we drop both objects. Would the more massive object fall twice as
fast as the less massive object? Aristotle said the larger mass would move
faster. But surely you could test Aristotle's thinking by actually making
two balls, one twice as massive, and dropping them to see! This is what
Galileo did. You are probably familiar with his most famous version of this
experiment, dropping objects from the leaning tower of Pisa.\FRAME{dtbpFU}{%
2.1021in}{1.5797in}{0pt}{\Qcb{Leaning Tower of Pisa where tradition has it
Galileo demonstrated his experiment to his colleagues.}}{}{Figure}{\special%
{language "Scientific Word";type "GRAPHIC";maintain-aspect-ratio
TRUE;display "USEDEF";valid_file "T";width 2.1021in;height 1.5797in;depth
0pt;original-width 9.9912in;original-height 7.491in;cropleft "0";croptop
"1";cropright "1";cropbottom "0";tempfilename
'OJ4NR308.wmf';tempfile-properties "XPR";}}

But really this was a demonstration for his colleagues at the university of
what Galileo had discovered by hours of careful measurement of objects
moving down ramps with little bells on them to provide feedback on the
ball's progress.

\FRAME{dtbpFU}{3.1151in}{2.3425in}{0pt}{\Qcb{Galileo's Apparatus for testing
Aristotle's ideas of mass and motions.}}{}{Figure}{\special{language
"Scientific Word";type "GRAPHIC";maintain-aspect-ratio TRUE;display
"USEDEF";valid_file "T";width 3.1151in;height 2.3425in;depth
0pt;original-width 4.241in;original-height 3.1834in;cropleft "0";croptop
"1";cropright "1";cropbottom "0";tempfilename
'OJ4NTG09.wmf';tempfile-properties "XPR";}}

Aristotle's idea was proven wrong by trying the idea on real objects. The
idea predicted an outcome, and the outcome did not happen when tested.
Science (and physics) was born!

\subsection{The Big Idea from Physics used in Leadership}

It is only natural that a big idea that worked so well in one area should
find itself in another. Testing a theory by collecting evidence is an
approach to decision making that is used in leadership today. If you are
trained in management you might remember the words \textquotedblleft
Management by Evidence,\textquotedblright\ or you might have observed the
current trend towards \textquotedblleft big data efforts.\textquotedblright\
The idea behind both is constant feedback to see if things are working as
predicted. Although many have written on this topic, we will give only one
comment, from an article by Jeffrey Pfeffer and Robert I. Sutton:

\begin{quotation}
Evidence-based management -- the notion that real knowledge in the form of
empirical analysis of results is the shortest path to the best business
decisions... From our research, we are convinced that when companies base
decisions on evidence, they enjoy a competitive advantage. \cite%
{PfefferandSutton}
\end{quotation}

From the engineering world, the ISO9000 standards incorporate the big idea

in standard practices.\cite{Hooper}

\FRAME{dtbpF}{3.915in}{2.9352in}{0in}{}{}{Figure}{\special{language
"Scientific Word";type "GRAPHIC";maintain-aspect-ratio TRUE;display
"USEDEF";valid_file "T";width 3.915in;height 2.9352in;depth
0in;original-width 3.8666in;original-height 2.8911in;cropleft "0";croptop
"1";cropright "1";cropbottom "0";tempfilename
'OJ4NXZ0A.wmf';tempfile-properties "XPR";}}

Notice the emphasis on feedback and testing. Objectives are defined up front
(3, and 4) and tested (5). Where the objectives are not met, the ideas (the
processes) are changed based on the outcome of the test.

\subsection{Not just any data}

Back in the 80's \textquotedblleft Management by Fact\textquotedblright\ was
a popular attempt to bring the principals of science into leadership. But
its implementation was often fundamentally flawed. The \textquotedblleft
facts\textquotedblright\ were often \textquotedblleft
metrics\textquotedblright\ designed to make leaders look good. When one of
us was at Eastman Kodak, we implemented Management by Fact in our IT
department. The IT leadership picked a metric---the number of closed
tickets. A \textquotedblleft ticket\textquotedblright\ was created every
time an employee called the IT department to report a computer problem.
Tragically, closing tickets was not the actual objective of the IT
department. The goal was to fix computer problems. You can guess what
happened. Soon the IT personnel were closing tickets no matter what
happened. Your hard drive crashed, they replaced it and closed the ticket.
But they did not recover the data that was on the defunct hard drive, data
that was necessary for the business to function.. Your internet service
stopped working. They logged the complaint, then closed the ticket. The
actual work of the company was adversely affected, at times terribly so,
because few computer problems were actually solved. Lost data and logged
problems don't make a business run. A true implementation of the ideas of
science would mean distilling leadership ideas into a series of actions with
predicted outcomes that further the goals of the organization. Then the
desired outcomes must be compared to the actual outcomes, not artificial
metrics, to determine if the leadership ideas have been successful. One of
the key characteristics of physics is that prediction is nearly everything,
explaining past behavior is only part of the program. In leadership, this
predictive skill is essential as well. In fact, hindsight is not really
20/20 unless the insight gained from looking back leads to new ideas that
better predict outcomes. And the key to this improvement is feedback. It is
the feedback that makes physics, physics.

Over the past century a youth organization known as the The Boy Scouts of
America was in the business of teaching leadership. The Boy Scouts had a
saying \textquotedblleft feedback is a gift.\textquotedblright\ The Boy
Scouts seemd to be an example of implementing this big idea from physics.

\subsection{Alma and the Big Idea}

It is interesting that we see this big idea in the Book of Mormon. Alma
tells us:

\begin{quotation}
Now, if ye give place, that a seed may be planted in your heart. . . it will
begin to swell within your breasts; and when you feel these swelling
motions, ye will begin to say within yourselves--It must needs be that this
is a good seed. . . But behold, as the seed swelleth, and sprouteth, and
beginneth to grow, then you must needs say that the seed is good; for behold
it swelleth, and sprouteth, and beginneth to grow. And now, behold, will not
this strengthen your faith?
\end{quotation}

The process of gaining a testimony seems to involve this same need for
feedback and evaluation.

\section{Predictable Unpredictably}

A year after Galileo died, Isaac Newton was born. Newton used the method
developed by Galileo to do great work in furthering our understanding of the
universe. Newton viewed the universe as a big machine. He thought that to
understand the whole universe you should start by examining and
understanding its parts. This idea is now called \textquotedblleft
reductionism.\textquotedblright\ For Newton the universe was like a
clockwork. It was predictable. If you understood each cog and fly wheel, you
understood the whole clock.\FRAME{dtbpFU}{3.4468in}{2.5864in}{0pt}{\Qcb{Sir
Issac Newton and an implementation of a \textquotedblleft
clockwork\textquotedblright\ universe model.}}{}{Figure}{\special{language
"Scientific Word";type "GRAPHIC";maintain-aspect-ratio TRUE;display
"USEDEF";valid_file "T";width 3.4468in;height 2.5864in;depth
0pt;original-width 4.3171in;original-height 3.2335in;cropleft "0";croptop
"1";cropright "1";cropbottom "0";tempfilename
'OJ4OEC0D.wmf';tempfile-properties "XPR";}}

This reductionist viewpoint has provided great success. By reducing complex
systems to parts, and using the big idea to test our understanding, we put
man on the moon! The industrial revolution was fueled by this idea. For
example, Model-T Ford cars could be built quickly and cheaply by structuring
the manufacturing organization around the collection of parts of the car.
The reductionist ideas of Newton were bound to make their way into
leadership. Our traditional management structure is based on this approach.
It is characterized by

\begin{itemize}
\item Rigid structure

\item Tightly defined roles

\item Many levels of leadership (like engineered systems and subsystems)

\item Thinking of people as part of the organizational \textquotedblleft
machine\textquotedblright
\end{itemize}

This reductionist view point is also at the heart of the deterministic model
in organizational psychology. Despite this great success of reductionism
there were some physics problems that did not seem to lend themselves easily
to this approach. But generally these \textquotedblleft
outliers\textquotedblright\ could be ignored. So much progress was being
made that these special cases could be ignored. But in 1944, one of these
special problems, weather prediction, was about to become important.

\subsection{Transition to Chaos}

Scientists and engineers were enamored with the regularity of the Newtonian
reductionist universe, and who can blame them? The approach gave us space
flight, air flight, Microwave ovens, etc. So we ignored and avoided cases
where predictable behavior was not possible. The result was that everything
seemed predictable, since we only paid attention to the predictable part of
the world.

Weather prediction was one of these unpredictable cases. The agriculture
industry had an interest in being able to predict the weather, but weather
had not proven solvable with a reductionist view point. But most scientists
weren't farmers, so this case was ignored for areas that showed more easy
promise. But in 1944, the middle of World War II, the Pacific Fleet was
caught unaware by Typhoon Cobra \FRAME{dtbpFU}{3.5561in}{2.7008in}{0pt}{\Qcb{%
USS Cowpens (CVL-25) During Typhoon Cobra}}{}{Figure}{\special{language
"Scientific Word";type "GRAPHIC";maintain-aspect-ratio TRUE;display
"USEDEF";valid_file "T";width 3.5561in;height 2.7008in;depth
0pt;original-width 9.9912in;original-height 7.5749in;cropleft "0";croptop
"1";cropright "1";cropbottom "0";tempfilename
'OJ4ON70E.wmf';tempfile-properties "XPR";}}

Seven hundred and ninety officers and sailors lost their lives in the storm.
The USS Hull, USS Monaghan, USS Spence all sank. 146 Aircraft were lost, and
all other vessels were damaged, some severely.\cite{Chen} Admiral Nimitz was
directing the fleet when the typhoon struck. The Admiral demanded we never
be caught blind by weather events again. He wanted this difficult,
non-predictable problem solved. This was the beginning of the National
Weather Service. But weather is a non-Newtoninan system. We could not ignore
the non-predictable part of the universe any more!

\subsection{Chaos, a new way to look at things}

Edward Lorenz took up the challenge of predicting the weather. He noticed
that in his calculations, he could do exactly\ the same calculation with
exactly the same inputs, but the computer would give different results every
time. \FRAME{dtbpFU}{4.8199in}{1.9567in}{0pt}{\Qcb{Lorenz and his Weather
Prediction Calculations (from Lorentz's 1961 printouts)}}{}{Figure}{\special%
{language "Scientific Word";type "GRAPHIC";maintain-aspect-ratio
TRUE;display "USEDEF";valid_file "T";width 4.8199in;height 1.9567in;depth
0pt;original-width 6.2587in;original-height 2.5253in;cropleft "0";croptop
"1";cropright "1";cropbottom "0";tempfilename
'OJ4OQT0F.wmf';tempfile-properties "XPR";}}

The problem got worse as he tried to predict farther into the future. What
he discovered was that the weather patterns depended on even slight
differences in the starting point for the calculations. The calculations
were so sensitive, that it was impossible to ever repeat the same prediction
twice. This startling realization meant that long-term weather prediction
was impossible---not what the Admiral wanted to hear! The problem being that
conditions today depend on conditions in the past week, month, year, decade,
etc. And the dependency is so intricate, that tiny, imperceptible changes
would radically change the outcome in the future. These initial conditions
are unknowable. So the system is not deterministic---we can't accurately
predict what will happen.

Lorenz and others decided to study simpler systems that had the same sort of
non-predictable behavior. One that was known was the chaotic water wheel. It
looks like this

\FRAME{dtbpF}{1.2258in}{1.6161in}{0pt}{}{}{Figure}{\special{language
"Scientific Word";type "GRAPHIC";maintain-aspect-ratio TRUE;display
"USEDEF";valid_file "T";width 1.2258in;height 1.6161in;depth
0pt;original-width 1.759in;original-height 2.3255in;cropleft "0";croptop
"1";cropright "1";cropbottom "0";tempfilename
'OJ4OW40H.wmf';tempfile-properties "XPR";}}

A wheel has buckets attached to it. Water falls from above and the wheel
turns as the buckets fill. \FRAME{dtbpF}{1.1167in}{1.6232in}{0pt}{}{}{Figure%
}{\special{language "Scientific Word";type "GRAPHIC";maintain-aspect-ratio
TRUE;display "USEDEF";valid_file "T";width 1.1167in;height 1.6232in;depth
0pt;original-width 1.4416in;original-height 2.1084in;cropleft "0";croptop
"1";cropright "1";cropbottom "0";tempfilename
'OJ4OY50I.wmf';tempfile-properties "XPR";}}

But the buckets are leaky. Because the buckets are heavy with water, the
wheel turns. But each bucket loses water, until it is empty when it gets to
the top. In this way, the wheel turns only one direction. But if the water
is forced downward, say, through a nozzle, the buckets are pushed away by
the fast moving water before they are full. The buckets can lose water too
quickly because they were never full. If the buckets are empty before they
reach the bottom, the wheel might reverse direction.\FRAME{dtbpF}{1.0786in}{%
1.5504in}{0pt}{}{}{Figure}{\special{language "Scientific Word";type
"GRAPHIC";maintain-aspect-ratio TRUE;display "USEDEF";valid_file "T";width
1.0786in;height 1.5504in;depth 0pt;original-width 1.4745in;original-height
2.1335in;cropleft "0";croptop "1";cropright "1";cropbottom "0";tempfilename
'OJ4OYX0J.wmf';tempfile-properties "XPR";}}

As the water flow rate increases, the reversals happen more often. The
Newtoninan prediction for the slow water case could be represented in a
graph like this. \FRAME{dtbpF}{3.4814in}{1.4085in}{0pt}{}{}{Figure}{\special%
{language "Scientific Word";type "GRAPHIC";maintain-aspect-ratio
TRUE;display "USEDEF";valid_file "T";width 3.4814in;height 1.4085in;depth
0pt;original-width 4.3932in;original-height 1.7598in;cropleft "0";croptop
"1";cropright "1";cropbottom "0";tempfilename
'PA83V400.wmf';tempfile-properties "XPR";}}

The graph collects all the characteristics of how the wheel turns into a
single point. In our case, we would predict, based on the water flow rate,
that the wheel would either turn to the right or to the left, and it would
turn at a particular speed based on the water flow rate. The idea
\textquotedblleft turns left at a particular speed\textquotedblright\ is
represented by a point on the left.

Now you could envision disturbing the wheel. You might grab the side and
slow it down temporarily, but a small disturbance will be short lived, and
the wheel will tend to return to it's particular speed and direction. In
physics, we say that the preferred speed and direction point is an
\textquotedblleft attractor\textquotedblright\ for the system.

\FRAME{dtbpF}{3.6943in}{1.6436in}{0pt}{}{}{Figure}{\special{language
"Scientific Word";type "GRAPHIC";maintain-aspect-ratio TRUE;display
"USEDEF";valid_file "T";width 3.6943in;height 1.6436in;depth
0pt;original-width 4.6673in;original-height 2.0605in;cropleft "0";croptop
"1";cropright "1";cropbottom "0";tempfilename
'PA83XN01.wmf';tempfile-properties "XPR";}}

As the water speed increases, the wheel will change direction at times. It
would slow down, stop and turn around the other way. We could represent this
as shown in the next figure.\FRAME{dtbpF}{3.6029in}{1.6498in}{0pt}{}{}{Figure%
}{\special{language "Scientific Word";type "GRAPHIC";maintain-aspect-ratio
TRUE;display "USEDEF";valid_file "T";width 3.6029in;height 1.6498in;depth
0pt;original-width 4.6061in;original-height 2.0942in;cropleft "0";croptop
"1";cropright "1";cropbottom "0";tempfilename
'PA840O03.wmf';tempfile-properties "XPR";}}

Notice that we picked up an additional attractor. As the flow rate increases
even more, the wheel will change directions more often. \FRAME{dtbpF}{3.453in%
}{1.6001in}{0pt}{}{}{Figure}{\special{language "Scientific Word";type
"GRAPHIC";maintain-aspect-ratio TRUE;display "USEDEF";valid_file "T";width
3.453in;height 1.6001in;depth 0pt;original-width 4.4624in;original-height
2.0516in;cropleft "0";croptop "1";cropright "1";cropbottom "0";tempfilename
'PA842604.wmf';tempfile-properties "XPR";}}But what Lorenz and his
colleagues found was that above a particular flow rate, the behavior of the
wheel was no longer predictable.\FRAME{dtbpF}{4.2469in}{2.3603in}{0pt}{}{}{%
Figure}{\special{language "Scientific Word";type
"GRAPHIC";maintain-aspect-ratio TRUE;display "USEDEF";valid_file "T";width
4.2469in;height 2.3603in;depth 0pt;original-width 5.2225in;original-height
2.8889in;cropleft "0";croptop "1";cropright "1";cropbottom "0";tempfilename
'PA847L05.wmf';tempfile-properties "XPR";}}

The wheel still turned around, so it was not completely unpredictable. But
for any given time, they could not predict what direction the wheel would
go, nor could they predict the exact speed. This was like the weather
problem!

Notice that in our graph, the wheel never turns with our predicted attractor
speed and direction. The path never reaches the Newtonian points. But it is
also never all that far away from these two points.

\subsection{Human Chaos}

At this point, you may be thinking, \textquotedblleft Great! But I am a
leader, not a weather man!\textquotedblright\ But it turns out that humans
are made more like stormes than clocks.\cite{Sapolsky}\FRAME{dtbpFU}{1.9527in%
}{2.636in}{0pt}{\Qcb{Robert Sapolski}}{}{Figure}{\special{language
"Scientific Word";type "GRAPHIC";maintain-aspect-ratio TRUE;display
"USEDEF";valid_file "T";width 1.9527in;height 2.636in;depth
0pt;original-width 2.5918in;original-height 3.5085in;cropleft "0";croptop
"1";cropright "1";cropbottom "0";tempfilename
'OJ4P8B0P.wmf';tempfile-properties "XPR";}}

Robert Sapolsky, one of the leading neurobiologists of our time, tells us we
are non-Newtonian systems, very like weather or chaotic wheels. And there is
a good reason for this. Our design engineer was very frugal. For example,
think of blood flow. From the Aorta, our vessels must split to make smaller
vessels. This splitting happens over and over again until we have
capillaries that can feed our cells.\FRAME{dtbpFU}{3.1194in}{2.341in}{0pt}{%
\Qcb{Blood Vessel Bifurcation}}{}{Figure}{\special{language "Scientific
Word";type "GRAPHIC";maintain-aspect-ratio TRUE;display "USEDEF";valid_file
"T";width 3.1194in;height 2.341in;depth 0pt;original-width
9.9912in;original-height 7.4832in;cropleft "0";croptop "1";cropright
"1";cropbottom "0";tempfilename 'OJ4P9Z0Q.wmf';tempfile-properties "XPR";}}

If our bodies needed instruction on exactly how far to go before each vessel
splits into two, it would take millions more genes than we have to encode
the process. To be efficient, our design engineer gave up exact Newtonian
specifications. Instead we encode the pattern: Grow, then split. And we
repeat this pattern at every level down to capillaries. This is more
efficient for encoding instructions. But it means the exact specification of
each vessel is not known. But in the end, all that matters is that blood
flow happens and that all the parts of the body are nourished.\cite{Sapolsky}

Neurons in our brain follow the same pattern. Our brains are inherently not
Newtonian systems. The science developed by Lorenz and others is called the
science of Deterministic chaos. It says that things are sometimes not
predictable, but in predictable ways (we won't know which way the wheel goes
at any given time, but we know it turns some way).

Other physicists have done work in the realm of small particles that give
similar results. You may have heard of Heisenberg's uncertainty principle.
This principal states that there is a fundamental limit in what can be known.

\FRAME{dtbpFU}{2.3462in}{3.4515in}{0pt}{\Qcb{Werner Karl Heisenberg}}{}{%
Figure}{\special{language "Scientific Word";type
"GRAPHIC";maintain-aspect-ratio TRUE;display "USEDEF";valid_file "T";width
2.3462in;height 3.4515in;depth 0pt;original-width 5.0747in;original-height
7.491in;cropleft "0";croptop "1";cropright "1";cropbottom "0";tempfilename
'OJ4PCO0R.wmf';tempfile-properties "XPR";}}

Which of course, means there is a fundamental limit on what can be
predicted. Rodger Penrose, a coauthor with Stephen Hawking, theorizes that
the brain is such a predictably unpredictable system.\cite{Penrose} And if
the brain is unpredictable, behavior will be unpredictable.

An example of this, given by Sapolsky, is an experiment with territorial
fish. If ten of the fish are placed in a tank, they will establish a
dominance order. Early researchers theorized that if they put the fish, two
at a time, in a tank, and recorded which was dominant, that they could
predict the dominance order of all ten once placed in the communal tank.
This is a good reductionist prediction. You understand the parts, this time
of a society of fish, and you understand the whole.

The experiment was run. Not only was the predicted dominance order wrong, it
was terribly wrong. There was no correlation at all between the prediction
and the actual outcome. It seems that Sapolsky and Penrose are right. A
reductionist model is not so good when applied to living organisms and the
societies they build.\cite{Sapolsky}

Others have come to the conclusion that uncertainty and chaos in human
systems need to be addressed.\cite{Wheatley}\cite{Tobak} The question is,
can the science that tries to deal with weather and other chaotic systems be
used to deal with living things, in particular, people?

\section{Forecasting: How to predict the unpredictable}

In 2005, Hurricane Katrina hit land in New Orleans. This time it was no
surprise.

\FRAME{dtbpFU}{3.8173in}{2.5521in}{0pt}{\Qcb{Hurricane Katrina and Karina
Forecast.}}{}{Figure}{\special{language "Scientific Word";type
"GRAPHIC";maintain-aspect-ratio TRUE;display "USEDEF";valid_file "T";width
3.8173in;height 2.5521in;depth 0pt;original-width 4.9502in;original-height
3.3001in;cropleft "0";croptop "1";cropright "1";cropbottom "0";tempfilename
'OJ4PGR0S.wmf';tempfile-properties "XPR";}}

The NOAA weather service people plotted the course of the storm, and the
prediction was fairly accurate. What made the difference?

The answer was the physicist's friend, feedback!

The weather scientists knew the relationships between parts of the weather
system. They understood the chaotic nature of the weather. But to do
accurate short term predictions then needed constant feedback from the
actual weather system. In weather forecasting, this takes the form of
receiving new data on how the weather really happened every 6 hours. Current
weather scientists use a computer and the knowledge of how weather systems
relate to each other to forecast the near future, but they don't try to
predict too far. The system is chaotic. We can't make long-term predictions.
The scientists had to give up predicting the weather next year, and had to
be content to predict what will happen next week. But for the Navy, and
commercial shipping, that is good enough! So in the distant future we can
tell that there will be some sort of weather (it is not totally chaotic,
with no bounds to the behavior), but long term forecasts won't tell us what
weather will really happen. We did not find the Newtoninan attractor point,
but we did find a solution that met the actual objective, no more surprise
storms.

\FRAME{dtbpFU}{3.5025in}{2.629in}{0pt}{\Qcb{GOES Satellite and imagery,
Weather balloon, Defense Meteorological Satellite Program Satellite, Weather
Radar.}}{}{Figure}{\special{language "Scientific Word";type
"GRAPHIC";maintain-aspect-ratio TRUE;display "USEDEF";valid_file "T";width
3.5025in;height 2.629in;depth 0pt;original-width 4.0499in;original-height
3.0338in;cropleft "0";croptop "1";cropright "1";cropbottom "0";tempfilename
'OJ4PLT0T.wmf';tempfile-properties "XPR";}}

A business example of this would be the rise of Walmart. Walmart keeps track
of what customers buy (feedback that matters to the objective!) Walmart even
has an electronic system that tracks purchases and plans deliveries based on
what has actually been purchased. As trends change, the Walmart forecast
model adjusts to those trends, because it has a constant feed of pertinent
data. It won't predict that the world will go crazy to get a new version of
the pet rock. But once pet rocks start to sell, Walmart will be ready to
ship them. The feedback makes the difference. And Walmart is ready to ship
something. So all that is needed is the data to drive the near term forecast
of sales. Even then, a chaotic system likely won't bring the exact results
the leader may want. So what are we to do?

Again, Sapolsky gives a biological example. Suppose we inoculate ten lab
rats with a vaccine. Then we watch the reactions of the rats. We might find
that three respond better than we thought. Six respond as was expected. But
one does not respond at all. If the rats were Newtonian systems, we would
take apart the non responsive rat to find which rat part was at fault. But
this is like taking apart a cloud to find out why it did not rain. A rat
system is not Newtonian, it is not deterministic. So at some point, taking
apart the rat won't help. But in the mean time, we have benefited nine rats.%
\cite{Sapolsky} And that is the key.

Our Newtonian ideal may never be realized, in fact, not even approached! But
as long as leaders can change their thinking to accept a little uncertainty
in the outcomes, we can benefit those we lead.

Walmart, and Sapolsky have decided that the exact, deterministic result is
not so important. Rather, being in the ballpark, with an idea of what to do
next based on feedback may work better than any alternative.

\section{Clocks and People: Two different leadership styles}

So what does physics tells us about leadership? If you are making machines,
a reductionist vision of leadership might be fine. If you are not, then
there will be a cost in a reductionist leadership style. Part of that cost
will undoubtedly be frustration on the part of the leader because the
Newtonian ideal may not be obtainable no matter what you do. Even if you do
make machines, the workers are likely to dislike being treated as only parts
of an organizational machine, because they are not like cogs. They are
deterministically chaotic systems.

\FRAME{dtbpFU}{2.6556in}{1.6125in}{0pt}{\Qcb{Ford Assembly Line and Google
Office}}{}{Figure}{\special{language "Scientific Word";type
"GRAPHIC";maintain-aspect-ratio TRUE;display "USEDEF";valid_file "T";width
2.6556in;height 1.6125in;depth 0pt;original-width 4.7668in;original-height
2.8824in;cropleft "0";croptop "1";cropright "1";cropbottom "0";tempfilename
'OJ4Q6P0X.wmf';tempfile-properties "XPR";}}

As a young leader, one of us was faced with an unproductive employee. At the
time, this young leader viewed the world in a reductionist way, so the
thought was to replace the \textquotedblleft broken cog.\textquotedblright\
It turned out that it cost \$10,000 to replace one employee, and then when
the \textquotedblleft replacement\textquotedblright\ arrived there was a six
month training period before the new employee was productive! Replacing the
\textquotedblleft cog\textquotedblright\ was really expensive. Later as a
more experience manager, this same leader approached a very similar
situation by working with the employee to develop the skills they lacked to
keep themselves productive. This proved to be a successful, cost effective,
approach. And in a non-Newtonian fashion, it improved the moral of the
entire group to see that the leader cared about the individual as well as
the business objective.

This managed chaos approach seems to teach us that the relationship is all
important. In the weather system, the scientists needed to understand how
weather systems interact. Then feedback was useful in determining forecasted
behavior. Physicists are fond of saying that \textquotedblleft the numbers
aren't that important\textquotedblright\ but \textquotedblleft the
relationship is everything.\textquotedblright

A deterministically chaotic leadership style might include a less rigid
structure with less strictly defined roles. Google might be a good example.
The young Google organization functioned with teams and less rigid
structures. It even experimented with allowing employees to control what
they did for a part of their time. They did have to be productive, but there
was not a Newtonian type goal. Some of Google's favorite products (e.g.
gmail) were produced this way. Incidentally, Google has recently changed
their approach to be a little more Newtonian, but such innovation time is
still part of the culture. Another example comes from an experience at
Hughes Information Technology Systems. A group building a custom ground
station to control weather satellites had become badly behind schedule.
Another group was asked to step in and catch up to make the schedule. The
leaders knew that the group would need to work long hours and make many
sacrifices to get the job done on time. They also realized that if you push
too hard, like with our chaotic wheel, you may get chaotic results. So they
built into their plan constant feedback on how the employees were feeling.
They helped them manage and balance their family needs and responsibilities,
provided rewards for milestones achieved---not promotions---but meaningful
representations of gratitude. By being in touch with what the employees were
experiencing and responding to enable the employees, the project was on
time, and the customer was very impressed. More work came from this success.
And the team as well as the leadership were all happy with the experience.%
\FRAME{dtbpF}{4.1599in}{3.0654in}{0pt}{}{}{Figure}{\special{language
"Scientific Word";type "GRAPHIC";maintain-aspect-ratio TRUE;display
"USEDEF";valid_file "T";width 4.1599in;height 3.0654in;depth
0pt;original-width 3.5829in;original-height 2.6333in;cropleft "0";croptop
"1";cropright "1";cropbottom "0";tempfilename
'OJ4QMT12.wmf';tempfile-properties "XPR";}}

Of course, the idea of managed chaos as a leadership style is nothing new.
The Mormon Scholar Hugh Nibley attributed our (the Allied) success in World
War II to our ability to manage chaos.

\FRAME{dtbpFU}{2.0631in}{1.6232in}{0pt}{\Qcb{US Soldiers at the Battle of
the Bulge}}{}{Figure}{\special{language "Scientific Word";type
"GRAPHIC";maintain-aspect-ratio TRUE;display "USEDEF";valid_file "T";width
2.0631in;height 1.6232in;depth 0pt;original-width 6.8in;original-height
5.3419in;cropleft "0";croptop "1";cropright "1";cropbottom "0";tempfilename
'OJ4Q1T0V.wmf';tempfile-properties "XPR";}}

The German troops were mechanized, like clock works. But that meant that
even when they knew the Normandy invasion was in progress, they would not
move forward without orders. The American and British forces had the same
goals, to win! But the style of leadership was different. The Americans gave
lower ranking officers the authority to get the job done and protect their
troops in whatever way seemed best within their unit goal.\cite{Nibley}
Nibley observed that this made all the difference.\FRAME{dtbpFU}{3.4575in}{%
1.3083in}{0pt}{\Qcb{Mozart and Count Basie, two different styles.}}{}{Figure%
}{\special{language "Scientific Word";type "GRAPHIC";maintain-aspect-ratio
TRUE;display "USEDEF";valid_file "T";width 3.4575in;height 1.3083in;depth
0pt;original-width 4.958in;original-height 1.8585in;cropleft "0";croptop
"1";cropright "1";cropbottom "0";tempfilename
'OJ4Q4A0W.wmf';tempfile-properties "XPR";}}

Let's go back to thinking about CERN and remember the rooms full of
computers to process the data, the natural feedback from the experiment. The
idea of management by feedback is central to modern physics. A colleague of
mine, a physicist, listened patiently to me explain this shift in the
physics model behind leadership, and summed it up succinctly as
\textquotedblleft so you want us to play jazz instead of
Mozart.\textquotedblright\ Perhaps this description is best. Mozart is
predictable, even mathematical. Jazz is improvisational and unpredictable.
Both styles have their place! But a jazz ensemble must trust and rely on
each other. The relationships are key, the product uncertain, undetermined,
and never exactly the same twice, but always good because the musicians
listen to and react to applause as well as each other. That is managed chaos.

\bibliographystyle{IEEEtran}
\bibliography{Leadership}

\end{document}
